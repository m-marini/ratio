\section{Autovalori e autovettori}
Sia ora $\vec{e'_i}$ una base formata dagli autovettori della matrice di
trasformazione $A$.
Abbiamo che
\begin{equation}
  \vec{e'_i}=\vec{e_j}b_{ji}
  \label{equno:autovalori.1}
\end{equation}

Le equazioni (\ref{equno:autovalori.1}) devono soddisfare le equazioni per
gli autovettori, quindi
\begin{equation}
  \sum_{k=1}^n\left( a_{jk} - \lambda_i \delta_{jk}\right)b_{ki}=0
  ,\;\;\;\;i,j=1\dots n
  \label{equno:autovalori.2}
\end{equation}

Poniamo ora $b_{ni}=1$, la matrice di trasformazione \`e quindi
\begin{displaymath}
  B=
  \left\Vert
  \begin{array}{lll}
    b_{11}&\dots&b_{1n}
    \\
    \dots&\dots&\dots
    \\
    b_{(n-1)1}&\dots&b_{(n-1)n}
    \\
    1&1&1
  \end{array}
  \right\Vert
\end{displaymath}
Il sistema (\ref{equno:autovalori.2}) \`e di rango $n-1$
quindi possima elimiare l'ultima riga e otteniamo
\begin{displaymath}
  \sum_{k=1}^{n-1}\left( a_{jk} - \lambda_i \delta_{jk}\right)b_{ki}m+a_{jn}=0
  ,\;\;\;\;i=1\dots n
  ,\;\;\;\;j=1\dots (n-1)
\end{displaymath}
\begin{equation}
  \sum_{k=1}^{n-1}\left( a_{jk} - \lambda_i \delta_{jk}\right)b_{ki}=-a_{jn}
  ,\;\;\;\;i=1\dots n
  ,\;\;\;\;j=1\dots (n-1)
  \label{equno:autovalori.3}
\end{equation}
in forma matriciale
\begin{equation}
  \left( A_{n-1} - \lambda_i \delta \right) B_i=A_n
  \label{equno:autovalori.4}
\end{equation}
dove
\begin{displaymath}
  A_{n-1}=
  \left\Vert
  \begin{array}{lll}
    a_{11}&\dots&a_{1(n-1)}
    \\
    \dots&\dots&\dots
    \\
    a_{(n-1)1}&\dots&a_{(n-1)(n-1)}
  \end{array}
  \right\Vert
  ,\;\;\;\;
  B_i = 
  \left\Vert
  \begin{array}{l}
    b_{1i}
    \\
    \dots
    \\
    b_{(n-1)i}
  \end{array}
  \right\Vert
  ,\;\;\;\;
  A_n = 
  \left\Vert
  \begin{array}{l}
    -a_{1n}
    \\
    \dots
    \\
    -a_{(n-1)n}
  \end{array}
  \right\Vert
\end{displaymath}
La (\ref{equno:autovalori.4}) ci permette di calcolari $Bi$ quindi
\begin{equation}
   B_i=\left( A_{n-1} - \lambda_i \delta \right)^{-1}A_n
  \label{equno:autovalori.5}
\end{equation}

Problema.

Calcoliamo la matrice di trasformazione della base in un piano $n=2$.

Avremo che
\begin{displaymath}
  A=
  \left\Vert
  \begin{array}{ll}
    a_{11}&a_{12}
    \\
    a_{21}&a_{22}
  \end{array}
  \right\Vert
\end{displaymath}
quindi
\begin{displaymath}
  A_1=a_{11}
  ,\;\;\;\;
  A_2=-a_{12}
  ,\;\;\;\;
  B_i=
  \left\Vert
  \begin{array}{l}
    -\frac{a_{12}}{a_{11} - \lambda_1}
    \\
    -\frac{a_{12}}{a_{11} - \lambda_2}
  \end{array}
  \right\Vert
  ,\;\;\;\;
  B=
  \left\Vert
  \begin{array}{ll}
    -\frac{a_{12}}{a_{11} - \lambda_1}&-\frac{a_{12}}{a_{11} - \lambda_2}
    \\
    1&1
  \end{array}
  \right\Vert
\end{displaymath}

La nuova base \`e quindi espressa dalle equazioni
\begin{displaymath}
  \begin{array}{l}
    \vec{e'_1}=\vec{e_1}\left(-\frac{a_{12}}{a_{11}-\lambda_1}\right)
    +\vec{e_2}
    \\
    \vec{e'_2}=\vec{e_1}\left(-\frac{a_{12}}{a_{11}-\lambda_2}\right)
    +\vec{e_2}
  \end{array}
\end{displaymath}

