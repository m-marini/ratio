\item \section{Trasformazione di una matrice da una base ad un altra}

Sia $\vec{x}=x_i \vec{e_i}=x_1 \vec{e_1}+\dots+x_n\vec{e_n}$ un vettore arbitrario nella base $\vec{e_i}$.

Il vettore $\vec{x}$ si trasforma mediante la matrice $A$ nel vettore $\vec{y}=y_i \vec{e_i}=y_1 \vec{e_1}+\dots+y_n\vec{e_n}$.

In forma matriciale avremo:
\begin{equation}
  Y=AX
  \label{equno:cambiobase.1}
\end{equation}
dove
\begin{displaymath}
  X=
  \left\Vert
  \begin{array}{l}
    x_1\\ \dots\\x_n
  \end{array}
  \right\Vert
  \;\;\;
  Y=
  \left\Vert
  \begin{array}{l}
    y_1\\ \dots\\y_n
  \end{array}
  \right\Vert
  \;\;\;
  A=
  \left\Vert
  \begin{array}{lll}
    a_{11}&\dots&a_{1n}\\
    \dots&\dots&\dots\\
    a_{n1}&\dots&a_{nn}
  \end{array}
  \right\Vert
\end{displaymath}

Introduciamo nello spazio una nuova base $\vec{e'_i}$ legata alla base $\vec{e_i}$ dalla formula
\begin{equation}
  \vec{e'_i}=\vec{e_j} b_{ji}  
  \label{equno:cambiobase.2}
\end{equation}
ovvero
\begin{displaymath}
  \begin{array}{l}
    \vec{e'_1}=\vec{e_1} b_{11}+\dots+\vec{e_n} b_{n1}
    \\
    \dots
    \\
    \vec{e'_n}=\vec{e_1} b_{1n}+\dots+\vec{e_n} b_{nn}
  \end{array}
\end{displaymath}.

Supponiamo che il vettore $\vec{x}$ si scriva nella base $\vec{e'_i}$ come
\begin{displaymath}
  \vec{x'}=x'_i\vec{e'_i}=x'_1\vec{e'_1}+\dots+x'_n\vec{e'_n}
\end{displaymath}

Possiamo allora scrivere $x_i\vec{i}=x'_j\vec{e'_j}$ dove il secondo membro
\`e stata sostituita l'espressione (\ref{equno:cambiobase.1}).
Uguagliando i coefficenti dei vettori $\vec{e_i}$ a destra e a sinistra
otteniamo $x_i=b_{ij}x'_j$ ovvero
\begin{equation}
  X=BX'
  \label{equno:cambiobase.3}
\end{equation}
dove
\begin{displaymath}
  B=
  \left\Vert
  \begin{array}{lll}
    b_{11}&\dots&b_{1n}\\
    \dots&\dots&\dots\\
    b_{n1}&\dots&b_{nn}
  \end{array}
  \right\Vert
  \;\;\;
  X'=
  \left\Vert
  \begin{array}{l}
    x'_1\\ \dots\\x'_n
  \end{array}
  \right\Vert
\end{displaymath}

Similmente otteniamo che
\begin{equation}
  Y=BY'
  \label{equno:cambiobase.4}
\end{equation}

Sostituendo la (\ref{equno:cambiobase.3}) e la (\ref{equno:cambiobase.4}) alla (\ref{equno:cambiobase.1})
otteniamo
\begin{equation}
  BY'=ABX'
  \label{equno:cambiobase.5}
\end{equation}

Essendo la $B$ una trasformazione di base ammette l'inversa $B^{-1}$.

Moltiplichiamo ambo i membri della (\ref{equno:cambiobase.5}) per  $B^{-1}$
\begin{displaymath}
  B^{-1}BY'=B^{-1}ABX'
\end{displaymath}
\begin{displaymath}
  Y'=B^{-1}ABX'
\end{displaymath}

Quindi la matrice di trasformazione $A$ si rappresenta nella nuova base con
\begin{equation}
  A'=B^{-1}AB
\end{equation}
