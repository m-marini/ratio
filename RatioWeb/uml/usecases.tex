\documentclass{article}
\begin{document}
  \title{RatioWeb\thanks{Versione \$Id: usecases.tex,v 1.2 2006/01/31 22:31:29 marco Exp $ $}}
  \author{Marco Marini}
  \maketitle
  \part{Casi d'uso}
  \section{Enter a value}
  Il caso d'uso permette di immetter una valore.
  
  \subsection{Scenario principale}
  \begin{enumerate}
    \item
      \label{EnterAValue.1}
      L'utente seleziona l'operazione di immisione di un valore che pu\'o
      essere: uno scalare, un vettore o una matrice.
      Nel caso di un vettore viene chiesto la dimensione del vettore, nel caso
      di una matrice vengono selezionate anche il numero di righe e colonne.
    \item
      \label{EnterAValue.2}
      Il sistema predispone la pagina con i campi di immissione.
      I campi sono: l'identificatore del valore (il nome della variabile) e i
      singoli scalari che compongono il vettore o la matrice.
    \item
      \label{EnterAValue.3}
      L'utente immette i valori nei campi in forma di espressioni.
    \item
      \label{EnterAValue.4}
      Il sistema valuta le espressioni e crea una variabile contenente il
      valore creato.
  \end{enumerate}
  
  \subsection{Errore di sintassi nei valori}
  Al punto \ref{EnterAValue.4} il sistema rileva degli errori di sintassi
  nell'espressione immessa dall'utente.
  \begin{enumerate}
    \item
      Il sistema predispone la stessa pagina di data-entry con l'evidenza
      degli errori riscontrati e il flussso riprende al punto
      \ref{EnterAValue.3}
  \end{enumerate}
  
  \subsection{Sostituzione della variabile}
  Al punto \ref{EnterAValue.1} e \ref{EnterAValue.2} l'utente sceglie di
  annullare il caso d'uso.
  \begin{enumerate}
    \item
      Il sistema cancella il valore e il flusso continua al punto \ref{EnterAValue.4}
  \end{enumerate}
  
  \subsection{Rinomina della variabile}
  Al punto \ref{EnterAValue.4} il sistema rileva che la variabile esiste gi\'a.
  \begin{enumerate}
    \item
      L'utente sceglie di rinominare la variabile.
    \item
      Il sistema predispone la stessa pagina di data-entry e il flusso riprende
      al punto \ref{EnterAValue.3}
  \end{enumerate}
  
  \subsection{Annullo caso d'uso}
  Al punto \ref{EnterAValue.1} e \ref{EnterAValue.3} l'utente sceglie di
  annullare il caso d'uso.
  \begin{enumerate}
    \item
      Il sistema riporta la navigazione alla pagina principale
  \end{enumerate}
  
\end{document}.

